%Modified from a template provided by Jennifer Pan, August 2011

\documentclass[10pt,letter]{article}
	% basic article document class
	% use percent signs to make comments to yourself -- they will not show up.
\usepackage{pdfsync}
\usepackage{amsmath}
\usepackage{amssymb}
\usepackage{amsthm}
	% packages that allow mathematical formatting

\usepackage{graphicx}
	% package that allows you to include graphics

\usepackage{setspace}
	% package that allows you to change spacing

\onehalfspacing
	% text become 1.5 spaced

\usepackage{fullpage}
% package that specifies normal margins

\usepackage[parfill]{parskip}

\newtheorem*{thm}{Theorem}
\newtheorem{nthm}{Theorem}

\begin{document}
	% line of code telling latex that your document is beginning

\title{Problem Set 2: Checkpoint}

\author{Katherine Cheng, Richard Davis}

% \date{Friday April 10, 2015}
	% Note: when you omit this command, the current date is automatically included
 
\maketitle 
	% tells latex to follow your header (e.g., title, author) commands.


\section*{Checkpoint Question: Recurrence Relations}

\paragraph{i)} 

\begin{thm}
  Given the recurrence relation
  \begin{align*}
    a_0 &= 1\\
    a_{n+1} &= 2a_n
  \end{align*}

  For any $n \in \mathbb{N}$, we have $a_n = 2^n$.
\end{thm}

\begin{proof}
  Let $P(n)$ be the statement ``any element of the recurrence relation is defined as $a_n = 2^n$.'' We will prove by induction that this holds for any natural number.

  For our base case, we need to show that $P(0)$ is true, meaning that $a_0 = 1 = 2^0$. Because $2^0 = 1$ we show that $P(0)$ is true.

  For the inductive step, assume that for some $k \in \matbb{N}$ that $P(k)$ holds, meaning that
  \begin{equation} \label{eq:squares}
    a_{k} = 2a_{k-1} = 2^k
  \end{equation}
  We need to show that $P(k+1)$ holds, meaning that $a_{k+1} = 2^{k+1}$. According to the recurrence relation, to get $a_{k+1}$ we do the following:
  \begin{align*}
    a_{k+1} &= 2a_k\\
    &= 2(2a_{k-1})\\ 
    &= 2(2^k) \quad (\text{via } \ref{eq:squares})\\
    &= 2^{k+1}
  \end{align*}
\end{proof}

\paragraph{ii)} Consider the following two recurrence relations:
\begin{align*}
  b_0 &= 1 &\quad c_0 &= 1\\
  b_{n+1} &= 2b_n - 1 &\quad c_{n+1} &= 2c_n + 1
\end{align*}
We will show that $b_n = 1$ and $c_n = 2^{n+1} - 1$.

\begin{thm}
  For any $n \in \mathbb{N}$, we have $b_n = 1$.
\end{thm}
\begin{proof}
  Let $P(n)$ be the statement ``any element of the recurrence relation is defined as $a_n = 1$.'' We will prove by induction that this holds for any natural number.
  
  For our base case we need to show that $P(0) = 1$, and this is true by definition.

  For the inductive step, assume that for some $k \in \matbb{N}$ that $P(k)$ holds, meaning that
  \begin{equation} \label{eq:triv}
    b_{k} = 2b_{k-1} - 1 = 1.
  \end{equation}
  We need to show that $P(k+1)$ holds, meaning that $b_{k+1} = 1$. According to the recurrence relation, to get $b_{k+1}$ we do the following:
  \begin{align*}
    b_{k+1} &= 2b_k - 1\\
    &= 2(2b_{k-1}) - 1 \\
    &= 2(1) - 1\quad (\text{via } \ref{eq:triv})\\
    &= 1
  \end{align*}
\end{proof}

\begin{thm}
  For any $n \in \mathbb{N}$, we have $c_n = 2^{n+1} - 1$.
\end{thm}
\begin{proof}
  Let $P(n)$ be the statement ``any element of the recurrence relation is defined as $c_n = 2^{n+1} - 1$.'' We will prove by induction that this holds for any natural number.
  
  For our base case we need to show that $P(0) = 1$. $P(0) = 2^{0 + 1} - 1 = 1$. 

  For the inductive step, assume that for some $k \in \matbb{N}$ that $P(k)$ holds, meaning that
  \begin{equation} \label{eq:fin}
    c_{k} = 2c_{k-1} + 1 = 2^{k+1} - 1.
  \end{equation}
  We need to show that $P(k+1)$ holds, meaning that $c_{k+1} = 2^{k+2} - 1$. According to the recurrence relation, to get $c_{k+1}$ we do the following:
  \begin{align*}
    c_{k+1} &= 2c_k + 1 \\
            &= 2(2^{k+1} - 1) + 1 \\
            &= 2^{k+2} - 2 + 1\quad (\text{via } \ref{eq:fin})\\\\
            &= 2^{k+2} - 1
  \end{align*}
\end{proof}

% \section*{Appendix: Referencing Equations}
% \begin{equation} \label{eq:divbyzero}
%   \frac {1} {0}
% \end{equation}

% This references \ref{eq:divbyzero}.

\end{document}
	% line of code telling latex that your document is ending. If you leave this out, you'll get an error

%%% Local Variables:
%%% mode: latex
%%% TeX-master: t
%%% End:
